% do not change these two lines (this is a hard requirement
% there is one exception: you might replace oneside by twoside in case you deliver 
% the printed version in the accordant format
\documentclass[11pt,titlepage,oneside,openany]{book}
\usepackage{times}


\usepackage{graphicx}
\usepackage{latexsym}
\usepackage{amsmath}
\usepackage{amssymb}
\usepackage{hyperref}

\usepackage{ntheorem}

% \usepackage{paralist}
\usepackage{tabularx}

% this packaes are useful for nice algorithms
\usepackage{algorithm}
\usepackage{algorithmic}

% well, when your work is concerned with definitions, proposition and so on, we suggest this
% feel free to add Corrolary, Theorem or whatever you need
\newtheorem{definition}{Definition}
\newtheorem{proposition}{Proposition}


% its always useful to have some shortcuts (some are specific for algorithms
% if you do not like your formating you can change it here (instead of scanning through the whole text)
\renewcommand{\algorithmiccomment}[1]{\ensuremath{\rhd} \textit{#1}}
\def\MYCALL#1#2{{\small\textsc{#1}}(\textup{#2})}
\def\MYSET#1{\scshape{#1}}
\def\MYAND{\textbf{ and }}
\def\MYOR{\textbf{ or }}
\def\MYNOT{\textbf{ not }}
\def\MYTHROW{\textbf{ throw }}
\def\MYBREAK{\textbf{break }}
\def\MYEXCEPT#1{\scshape{#1}}
\def\MYTO{\textbf{ to }}
\def\MYNIL{\textsc{Nil}}
\def\MYUNKNOWN{ unknown }
% simple stuff (not all of this is used in this examples thesis
\def\INT{{\mathcal I}} % interpretation
\def\ONT{{\mathcal O}} % ontology
\def\SEM{{\mathcal S}} % alignment semantic
\def\ALI{{\mathcal A}} % alignment
\def\USE{{\mathcal U}} % set of unsatisfiable entities
\def\CON{{\mathcal C}} % conflict set
\def\DIA{\Delta} % diagnosis
% mups and mips
\def\MUP{{\mathcal M}} % ontology
\def\MIP{{\mathcal M}} % ontology
% distributed and local entities
\newcommand{\cc}[2]{\mathit{#1}\hspace{-1pt} \# \hspace{-1pt} \mathit{#2}}
\newcommand{\cx}[1]{\mathit{#1}}
% complex stuff
\def\MER#1#2#3#4{#1 \cup_{#3}^{#2} #4} % merged ontology
\def\MUPALL#1#2#3#4#5{\textit{MUPS}_{#1}\left(#2, #3, #4, #5\right)} % the set of all mups for some concept
\def\MIPALL#1#2{\textit{MIPS}_{#1}\left(#2\right)} % the set of all mips





\begin{document}

\pagenumbering{roman}
% lets go for the title page, something like this should be okay
\begin{titlepage}
	\vspace*{2cm}
  \begin{center}
   {\Large Project Outline\\}
   \vspace{2cm} 
   {Data Mining Team Project\\}
   \vspace{2cm}
   {presented by\\
    Natalie Buchner \\
    Ivan Karachunskiy \\
    Sophia Maier \\
    Timo Strauch\\
    %Matriculation Number 9083894\\
   }
   \vspace{1cm} 
   {submitted to the\\
    Data and Web Science Group\\
    Prof.\ Dr.\ Bizer\\
    University of Mannheim\\} \vspace{2cm}
   {24\textsuperscript{th} April 2016}
  \end{center}
\end{titlepage} 

% no lets make some add some table of contents
%\tableofcontents
%\newpage

%\listofalgorithms

%\listoffigures

%\listoftables

% evntuelly you might add something like this
% \listtheorems{definition}
% \listtheorems{proposition}

\newpage


% okay, start new numbering ... here is where it really starts
\pagenumbering{arabic}
\chapter{Project Proposal}

\section{Our problem}
\label{cha:problem}

Why are certain gastronomy facilities more popular and economically successful than others?  
Search and rating websites such as Trip advisor, yelp etc. present business profiles containing specific properties, such as location, price range, opening hours, cuisine, and user based rating. As a result these properties play a decisive role in the decision process of potential customers looking for a place to drink their coffee or have a nice meal. However, if you only take those rather general properties into account, many businesses seem similar while their economical success often differs significantly. As a consequence we wonder whether there could be other determining factors that make people choose one business over another and help businesses to retain those customers. \\
Some of these attributes can be found when taking a look at textual reviews where people state their opinion on certain extended characteristics of the business and often give a concluding rating (e.g. through stars at a scale of 1 to 5). The goal of this project is to extract new success aspects of businesses from reviews. These can be used to give a more sophisticated view on the restaurant, bar or caf�. \\
The knowledge concluded as well as the gathered insight into business success factors could be interesting for existing and new businesses as well as start-up consultants or even venture capitalists as it could serve as a guideline when wanting to upgrade their facility in order to attract and retain more customers. 
Based on our anticipated findings further analysis could be performed in order to gain a sophisticated overview regarding current market situation in general. As a result existing and potential future market trends as well as market niches could be identified. 


Retreive additional aspects that describe the restaurant, bar, cafe in more depth and place them into the facilitie's yelp prole. This could help yelp users to make a faster decision based on further attributes important to them.current market situation in general


\section{The Yelp Dataset} %What data will we use
\label{cha:yelp}


\subsection{Background} %where will we get it
\label{sec:background}


\subsection{Integration of JSON}
\label{sec:integration}


\section{Our approach} %How will you solve the problem
\label{cha:approach}

\subsection{Preprocessing}
\label{sec:preprocessing}


\subsection{Algorithms}
\label{sec:algorithms}


\section{Measure of Success}
\label{cha:success}


\section{Possible Results}
\label{cha:conclusion}
As project outcome we would like to identify additional success factors and attractive business features important to customers. This knowledge could firstly be used by placing additional attributes important to customers into the facility's yelp profile. By that a more in depth representation of businesses (restaurants, bars, cafes etc.) is obtained, enabling yelp users to make a faster decision based on further determining attributes.\\ 
Secondly, the gathered insight into business success factors could be interesting for existing and new businesses as well as start-up consultants or even venture capitalists, as it could serve as a guideline when wanting to upgrade their facility in order to attract and retain more customers.\\ 
Based on our anticipated findings, further analysis could be performed in order to gain a sophisticated overview regarding the current market situation in general. As a result existing and potential future market trends as well as market niches could be identified.\\ 


%\bibliographystyle{plain}
%\bibliography{proposal-ref}

\end{document}
